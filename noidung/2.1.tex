\section{Cân bằng nhiệt}
	Ta tưởng tượng là vách ngăn cho nhiệt truyền qua, không dịch chuyển và không thấm (tức không cho các hạt qua lại). Biến số duy nhất của entropy chỉ là nội năng.
	
	Từ tính cộng được của entropy ta có
	\begin{equation*}
		\sigma = \sigma_1 + \sigma_2
	\end{equation*}
	Giả thiết rằng hệ con 1 và hệ con 2 nằm cân bằng với nhau, tức hệ kín ở cân bằng. Khi đó $ \sigma $ đạt cực đại. Điều kiện để $ \sigma $ đạt cực đại là $ \dd{\sigma} = 0 $.
	\begin{align}
		\dd{\sigma} & = \dd{\sigma_1} + \dd{\sigma_2} = 0 \notag \\
		\dd{\sigma} & = \left(\dfrac{\partial \sigma_1}{\partial U_1}\right) \dd{U_1} + \left(\dfrac{\partial \sigma_2}{\partial U_2}\right) \dd{U_2} = 0
	\end{align}
	Vì nội năng của hệ kín là hằng số, nên
	\begin{equation*}
		\dd{U} = \dd{U_1} + \dd{U_2} = 0
	\end{equation*}
	Do đó
	\begin{equation*}
		\dd{\sigma} = \left[\left(\dfrac{\partial \sigma_1}{\partial U_1}\right) - \left(\dfrac{\partial \sigma_2}{\partial U_2}\right)\right] \dd{U_1} = 0
	\end{equation*}
	Vì biến thiên $ dU_1 $ có thể tùy ý, nên
	\begin{equation}\label{eq:2.3}
		\dfrac{\partial \sigma_1}{\partial U_1} = \dfrac{\partial \sigma_2}{\partial U_2}
	\end{equation}
	Ta dẫn ra đại lượng $ \theta $ theo định nghĩa
	\begin{equation}\label{eq:2.4}
		\dfrac{1}{\theta} = \left(\dfrac{\partial \sigma}{\partial U}\right)_{V, N}
	\end{equation}
	Điều kiện \eqref{eq:2.3} trở thành
	\begin{equation}\label{eq:2.5}
		\theta_1 = \theta_2
	\end{equation}
	Kết quả này có thể dễ dàng mở rộng cho hệ kín gồm số bất kỳ các hệ con nằm cân bằng với nhau.
	
	Đại lượng $ \theta $ được gọi là nhiệt độ thống kê của hệ. Như vậy, nếu hệ ở trạng thái cân bằng thì nhiệt độ thống kê của tất cả các phần của nó là như nhau, nghĩa là không đổi trên toàn hệ. Cũng như entropy, nhiệt độ là một đại lượng có tính chất thống kê thuần túy và chỉ có ý nghĩa đối với các vật vĩ mô. Đại lượng $ T $ liên hệ với $ \theta $ bởi hệ thức
	\begin{equation}\label{eq:2.6}
		\theta = k_B \cdot T
	\end{equation}
	Với $ k_B $ là hằng số Boltzmann ($ k_B = 1,38 \times 10^{-16} $) gọi là nhiệt độ tuyệt đối. Trong hệ đơn vị CGS, đơn vị của $ \theta $ là erg, đơn vị của $ T $ là $ K $.
	
	Tiếp theo ta giả sử hai hệ con không cân bằng với nhau, nghĩa là hệ kín không nằm trong trạng thái cân bằng. Các nhiệt độ $ \theta_1 $ và $ \theta_2 $ khác nhau. Giả sử $ \theta_2 > \theta_1 $. Theo thời gian giữa hai hệ con sẽ thiết lập sự cân bằng khi mà các nhiệt độ sẽ trở nên bằng nhau. Entropy $ \sigma $ của hệ kín sẽ tăng, nghĩa là $ \dd{\sigma} > 0 $, hay
	\begin{align*}
		& \left[\left(\dfrac{\partial \sigma_1}{\partial U_1}\right) - \left(\dfrac{\partial \sigma_2}{\partial U_2}\right)\right] \dd{U_1} > 0 \\
		& \left[\dfrac{1}{\theta_1} - \dfrac{1}{\theta_2}\right] \dd{U_1} > 0
	\end{align*}
	Vì $ \theta_2 > \theta_1 $ nên $ \left[\dfrac{1}{\theta_1} - \dfrac{1}{\theta_2}\right] > 0 $. Vì vậy $ \dd{U_1} > 0 $ và do đó $ \dd{U_2} = - \dd{U_1} < 0 $. Năng lượng của hệ con 1 tăng còn năng lượng hệ con 2 giảm. Năng lượng chuyển từ vật có nhiệt độ cao hơn sang vật có nhiệt độ thấp hơn.