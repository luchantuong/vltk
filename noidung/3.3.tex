\section{Năng lượng tự do Helmholtz $ F $}
	Theo định nghĩa hàm $ F $ bằng
	\begin{equation}\label{eq:3.18}
		F \equiv U - TS
	\end{equation}
	Ta có
	\begin{align}\label{eq:3.19}
		\dd{F}
		& = \dd{U} - T\dd{S} - S\dd{T} 
		= -S\dd{T} - P\dd{V} \\
		& = \left(\dfrac{\partial F}{\partial T}\right)_{V} \dd{T} + \left(\dfrac{\partial F}{\partial V}\right)_{T} \dd{V} \notag
	\end{align}
	Từ \eqref{eq:3.19} suy ra
	\begin{align}\label{eq:3.20}
		& F = F\left(T,V\right) \\
		& S = -\left(\dfrac{\partial F}{\partial T}\right)_{V} \\
		& P = -\left(\dfrac{\partial F}{\partial V}\right)_{T}
	\end{align}
	Một khi thể tích $ V $ và nhiệt độ $ T $ là hai biến số độc lập thì nên xác định hàm $ F $, nhờ nó áp suất $ P $ và entropy $ S $ tính dễ dàng.